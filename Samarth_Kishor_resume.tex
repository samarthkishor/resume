% Modified from Trey Hunner's template (https://github.com/treyhunner/resume)

\documentclass{resume}

\usepackage[left=0.5in,top=0.5in,right=0.5in,bottom=0.5in]{geometry}
\usepackage{hyperref}
\usepackage{xcolor}
\hypersetup{
    colorlinks=false,
    urlbordercolor=white,
}
\urlstyle{same}

\name{Samarth Kishor}
\address{804--300--6897 \\ samarthkishor1@gmail.com \\ samarth.me \\ github.com/samarthkishor}

\def\nameskip{\medskip}
\def\sectionskip{\medskip}

\begin{document}

\vspace{-1em}
\begin{rSection}{Experience}

  \begin{rSubsection}{Amazon Web Services}{August 2022 -- Present}{Applied Scientist}{Arlington, VA}
    \item Co-authored a paper and patent application for a technique to apply static analyses to automatically scale the verification of real-world security protocol implementations. Implemented the technique on a 100k+ LoC Go codebase.
    \item Designed and implemented \href{https://github.com/awslabs/ar-go-tools}{static analyses} (program slicing, information flow, constrained mutability) to verify security properties for the \href{https://github.com/aws/amazon-ssm-agent}{SSM Agent Go codebase} deployed on millions of hosts.
    \item Led the verification of safety and liveness properties of a distributed protocol used for authorization by the majority of services at AWS via first-order logic modeling (Ivy) as well as bounded model-checking (P).
  \end{rSubsection}

  \begin{rSubsection}{Amazon Web Services}{July 2021 -- August 2022}{Software Development Engineer}{Arlington, VA}
    \item Mentored new hires and interns in Go programming, compilers, and advanced testing techniques.
    \item Wrote an end-to-end data transformation compiler in Go for a JSON-like language used to transfer terabytes of customer data per day. Rigorously demonstrated the correctness of the implementation with property-based testing.
    \item Simplified the design of a concurrent process manager in Go to eliminate all known data races.
  \end{rSubsection}

  \begin{rSubsection}{Amazon Web Services}{May 2020 -- August 2020}{Software Development Engineer Intern}{Arlington, VA}
    \item Extended an existing customer-facing AWS service by implementing a new serverless component in Python using Lambda, S3, and Dynamo.
    \item Service made heavy use of concurrency and parallelism to save customers time and compute resources, and improved the performance of the existing process by over 200\%.
  \end{rSubsection}

\end{rSection}

\begin{rSection}{Education}

  \begin{rSubsection}{University of Virginia}{August 2027 -- May 2021}{B.A. Computer Science, Technology Entrepreneurship Minor}{Charlottesville, VA}
    \item \textbf{TA Experience:} Programming Languages (Head TA), Theory of Computation, Discrete Mathematics
  \end{rSubsection}

\end{rSection}

\begin{rSection}{Projects}

  \begin{rSubsection}{OCamLox}{January 2020 -- April 2020}{https://github.com/samarthkishor/crafting-interpreters}{Personal Project}
    \item Implemented an interpreter for a JavaScript-like programming language from the book \href{https://craftinginterpreters.com/}{\textit{Crafting Interpreters}}.
    \item The interpreter consists of a hand-rolled recursive-descent parser and tree-walking evaluator written in OCaml.
  \end{rSubsection}

  \begin{rSubsection}{TriTag}{September 2018}{\url{https://devpost.com/software/tritag}}{MedHacks}
    \item Group won a Top Ten in Category Award out of hundreds of participants.
    \item Created a modern, digital approach to Triage involving an iPad app and an Arduino-based identifier.
  \end{rSubsection}

\end{rSection}

\begin{rSection}{Leadership}

  \begin{rSubsection}{HooHacks}{Fall 2018 -- Current}{External Relations Committee Co-Chair (Elected Position)}{Charlottesville, VA}
    \item Raised over \$15,000 for the 2020 hackathon by working directly with sponsors such as Capital One and CoStar.
    \item Coordinated workshops for 50+ students and served as a point-of-contact for sponsors during the 2019 hackathon.
  \end{rSubsection}

  \begin{rSubsection}{University Salsa Club}{Fall 2017 -- Current}{Choreographer}{Charlottesville, VA}
    \item Choreographed dances for the Fall 2018 and Spring 2020 Salsa Showcase.
    \item Performed for events in the Charlottesville community such as the Virginia Film Festival and Pancakes for Parkinson's.
  \end{rSubsection}

\end{rSection}

\begin{rSection}{Skills}

  \begin{tabular}{ @{} >{\bfseries}l @{\hspace{5ex}} l }
    Programming Languages & \textit{Fluent}: Go, Python, Java \hspace{1.5ex}
                            \textit{Proficient}: JavaScript, OCaml \hspace{1.5ex}
                            \textit{Familiar}: Rust, Prolog\\
    Tools \& Technologies & Automated Reasoning (Static analysis, P, Ivy), AWS, Redis, GNU/Linux\\
    Human Languages       & \textit{Fluent}: English \hspace{1.5ex}
                            \textit{Proficient}: Spanish, Japanese \hspace{1.5ex}
                            \textit{Familiar}: Hindi, Urdu
  \end{tabular}

\end{rSection}

\end{document}

%%% Local Variables:
%%% mode: latex
%%% TeX-master: t
%%% End:
